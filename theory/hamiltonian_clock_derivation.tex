\documentclass[12pt]{article}
\usepackage[margin=1in]{geometry}
\usepackage{amsmath, amssymb}
\usepackage{authblk}
\usepackage{graphicx}
\usepackage{physics}

\title{Hamiltonian Clock Derivation}
\author{Wright-Shawn}
\date{Waveframe v2.0 | 2025-08-01}

\begin{document}

\maketitle

\section*{Abstract}
This document outlines the derivation of the Hamiltonian constraint in the Waveframe v2.0 model, where the scalar field $\phi$ defines the emergent time coordinate. We derive the cosmological Hamiltonian in minisuperspace and recast the Friedmann equation in $\phi$-time.

\section{Starting Point: Action}

We begin with the scalar field action minimally coupled to gravity:

\begin{equation}
S = \int d^4x \, \sqrt{-g} \left[ \frac{1}{2} g^{\mu\nu} \partial_\mu \phi \partial_\nu \phi - V(\phi) \right]
\end{equation}

\section{Homogeneous Cosmology Assumption}

Assume a spatially flat Friedmann-Robertson-Walker (FRW) metric:

\begin{equation}
ds^2 = -dt^2 + a(t)^2 d\vec{x}^2
\end{equation}

Assume also that $\phi = \phi(t)$ (homogeneous). The Lagrangian density reduces to:

\begin{equation}
\mathcal{L} = a^3 \left[ \frac{1}{2} \dot{\phi}^2 - V(\phi) \right]
\end{equation}

\section{Canonical Momentum}

The conjugate momentum to $\phi$ is:

\begin{equation}
\pi_\phi = \frac{\partial \mathcal{L}}{\partial \dot{\phi}} = a^3 \dot{\phi}
\end{equation}

\section{Hamiltonian Density}

The Hamiltonian density is given by:

\begin{align}
\mathcal{H} &= \pi_\phi \dot{\phi} - \mathcal{L} \\
&= a^3 \left[ \frac{1}{2} \dot{\phi}^2 + V(\phi) \right]
\end{align}

Using $\dot{\phi} = \pi_\phi / a^3$, we can express this as:

\begin{equation}
\mathcal{H} = \frac{1}{2a^3} \pi_\phi^2 + a^3 V(\phi)
\end{equation}

\section{Hamiltonian Constraint}

In general relativity, the Hamiltonian constraint enforces $\mathcal{H} = 0$ for a closed universe (no external energy sources). Thus:

\begin{equation}
\frac{1}{2a^3} \pi_\phi^2 + a^3 V(\phi) = 0
\end{equation}

This relates the field energy to the geometry (via $a$).

\section{Reinterpreting $\phi$ as Time}

We now choose $\phi$ as a clock variable and treat $a(\phi)$ as the dynamical degree of freedom.

Define an effective lapse function $\alpha(\phi)$ via:

\begin{equation}
ds^2 = -\alpha(\phi)^2 d\phi^2 + a(\phi)^2 d\vec{x}^2
\end{equation}

This implies:

\begin{equation}
\frac{da}{d\phi} = \frac{\dot{a}}{\dot{\phi}} = \frac{\dot{a}}{\alpha(\phi)}
\end{equation}

\section{Effective Friedmann Equation}

Solving the Hamiltonian constraint for $\dot{a}$ and re-expressing in terms of $\phi$, we find the analog of the Friedmann equation:

\begin{equation}
\left( \frac{1}{a} \frac{da}{d\phi} \right)^2 = \frac{1}{3 M_{\text{Pl}}^2 \alpha(\phi)^2} \left[ \frac{1}{2} + V(\phi) \right]
\end{equation}

This governs the evolution of $a(\phi)$ in scalar field time.

\section{Summary}

\begin{itemize}
  \item The Hamiltonian constraint relates field energy to spacetime curvature.
  \item By identifying $\phi$ as time, we derive a new evolution equation in $\phi$-space.
  \item The lapse $\alpha(\phi)$ encodes how field time maps to proper time.
  \item This provides a mathematical foundation for emergent-time cosmology.
\end{itemize}

\end{document}
